\chapter{はじめに}
%

この教材は筑波大学生物資源学類1年生のための物理学
の教材である。授業「物理学I」(奈佐原顕郎)でも使う。
まず, 本書の基本的な方針を示しておこう。

\begin{itemize}
\item 想定する学生は, 少なくとも高校の数学IIIレベルの数学力があり, かつ, 
中学理科の1分野を理解している人。高校「物理基礎」「物理学」は不問。
\item 数学の副教材として, 「ライブ講義 大学1年生のための数学入門」
(もしくはその前身の生物資源学類「数学リメディアル教材」)を使う。
\item 本書の文中に「数学の教科書」という言葉がよく出てくるが, それは具体的には↑この教材を意味する。
\item 物理学の理解に数学の学力は必須。「あったら良い」という程度でなく, 「必須」。
この教材では, 前半部では主に「数学III」までの高校数学を使うが, 
後半部ではさらに大学の数学(微分方程式・偏微分・全微分・外積など)
も使う。これらの数学は「大学1年生のための数学入門」で自主的に学んで
おくこと。「とりあえず物理を勉強してみて, 数学が必要になったら
そのとき数学を勉強しよう」という考えは, 残念ながら通用しないだろう。
\item 高校物理の履修は前提としないが, 中学理科(1分野)の物理(特に, 力, 運動, エネルギーの内容)は前提とする。その復習のために, まず以下の本を読んで欲しい:
\begin{itemize}
\item 関口知彦・鈴木みそ「マンガ 物理に強くなる ... 力学は野球よりやさしい」 講談社ブルーバックス
\end{itemize}
%\item また, 高校物理の参考書として山本・左巻「新しい高校物理の教科書」(講談社ブルーバックス)をお勧めする。
\item 物理実験の映像をたくさん見よう。物理学は机上の空論では
なく, 実際の自然現象を説明する理論だから, 実験を通して理解することが
必要。テレビ番組では以下がお勧め: 
\begin{itemize}
\item 「NHK高校講座 物理」
\item 「ピタゴラスイッチ」特にピタゴラ装置
\item 「大科学実験」
\item 「世界一受けたい授業」\\...特にでんじろう先生
\end{itemize}
「NHK高校講座 物理」「大科学実験」はネット上で見ることもできる。他にも, ネット上(YouTube等)に, 良い動画がたくさんある(「物理エンジン」で検索!)。楽しんで見よう!
\item この教材の誤植訂正情報は, 以下のウェブサイトまたはmanabaに掲載する:\\
http://pen.envr.tsukuba.ac.jp/lec/physics/
\item この教材の内容は, (基本法則や慣習的に理解されていること以外は)全てオリジナルである。
\item ただし, Richard Feynman "Lectures on Physics" Addison-Wesleyから影響を
受けている(岩波書店から「ファインマン物理学」という題で和訳が出ている)。ちなみに, 
この本の著者が書いた自伝エッセイ「ご冗談でしょうファインマンさん」は秀逸であり, 一読を薦める。
\item ベクトルは, 細字や「文字に上矢印」ではなく, 太字で書くこと。
\item $x$を$\chi$と書かないように。
\item 数値を求める問題では, (無次元量でない限り)数値に単位をつけること。
数値の有効数字は, 特に指定がなければ, 2桁でよろしい。
\item 各章末に問題の解答を載せた。ただし一部の問題については解答を省略した
(「解答略」とすら載っていないものもある)。そのような問題は, テキストをしっかり
読めば自然に解答が見つかるはず。
\end{itemize}
\mv

\begin{faq}{\small\textgt{物理って必要なんですか? あんまり役立ち
そうな気がしないのですが} ... 物理は「役に立つ科学の
ナンバーワン」です。理系の学問で, 物理学のお世話になっていないものは
何一つありません。物理は空気と同じようなもので, 気にしなければ気づきま
せんが, 常に我々のまわりにある大切な存在です。}\end{faq}

\begin{faq}{\small\textgt{私は物理学は苦手だし, 私の将来には
必要ないと思っていたのですが} ... そういうメッセージは要警戒です(笑)。
ものごとの重要性や意味は, それを理解し習得してはじめてわからる, 
という面があります。「物理なんか必要ないよ」と言う人の多くは, 
その人自身が物理を理解していないのです。}\end{faq}

\begin{faq}{\small\textgt{物理って, 覚えた公式に数字を入れるだけ
じゃないのですか?} ... ちがいます。物理学は実験事実から導かれた基本法則と
数学で組み立てられる, 精密で普遍的な理論体系です。世の中のものの成り立ちや
現象の仕組みを根本から説明するスーパーパワーです。「公式に数字を入れる」
のは, 物理学のごく一部に過ぎません。}\end{faq}

\begin{faq}{\small\textgt{でも, ボールとかバネとか, ものの動き
をあれこれ考えるだけでしょ?}
 ... 「ものの動き」は物理学の対象のごく一部なのですが, 「ものの動き」が
わかるだけでも凄いことですよ。地震も津波も台風も「ものの動き」です。
化学反応も分子や原子という「ものの動き」です。魚が水中を泳ぐのも, 鳥が
空を飛ぶのも「ものの動き」です。高校物理で出てきたボールの軌跡や
バネの振動は, そのような森羅万象の「ものの動き」を説明する, 最初の
例にすぎません。}\end{faq}

\begin{faq}{\small\textgt{高校でやった物理基礎は面白くなかったのですが}
 ... それはあなたの心がまだ物理学に向いていなかったからではないでしょうか。
教科書に太字で記された公式を丸暗記して, 解法のパターンを覚え, 
時間内に正確に解く, みたいな勉強なら, いくらやっても物理は楽しくなりません。
物理の理論が実際の現象にどのように対応するのかを自分の頭で考え, 
試し, 納得するまで自問自答する, という作業をすれば, あなたの心は物理学
に向かって開かれ, 物理学が面白くなります。}\end{faq}

\begin{faq}{\small\textgt{物理, 嫌いです。どうすれば好きになれますか?} ... 
数学もそうですが, まず, 先入観を捨てて素直になること。「問題が解けるかどうか」
ではなく, 「自分が本当に理解しているか」に関心を向けること。テキストを丁寧に
読んで考えること。自分で試すこと。身の回りの現象に応用して考えてみること。
一方, やみくもに暗記とか, テストやレポートで点をもらうことだけにこだわると, 
嫌いになります。}\end{faq}

\begin{faq}{\small\textgt{でも, テストやレポートで点がとれないと
成績が悪くなるし単位も貰えないじゃないですか。} ... 
成績や単位は, その学問を理解し, 習得したことの証明です。ちゃんと
わかっていないのに良い成績をとろうとするのは「ごまかし」じゃないでしょうか?}\end{faq}

\begin{faq}{\small\textgt{でも, 成績が悪かったり単位がとれないと, 就職や
進学で困ります} ... 仮に就職や進学でごまかせても, その後で
見抜かれますよ。最後に評価されるのは\underline{実力}だけです。}\end{faq}

\begin{faq}{\small\textgt{でも, やはり成績や単位が気になります}
 ... なら, 残念だけど今のあなたにはこの授業は向いていないかもしれないですね...。}\end{faq}

\begin{faq}{\small\textgt{高校で物理を勉強していないので不安です}
... あまり関係ありません。高校での履修でむしろ重要なのは数学IIIです。}\end{faq}

\begin{faq}{\small\textgt{物理は高校で挫折しましたが, 高校数学
は学びました。リベンジできますかね?}
... 高校数学を修めた今こそ, 物理にチャレンジ&リベンジする時です! 
実際, 高校物理の難しさの大部分は, ちゃんと数学を使わない(文科省の
縛りで使っちゃいけないことになっている)ことにあります。ちゃんと数学を
使えば「覚えること」は半分以下に減り, 見通しもすっきりします。}\end{faq}

\begin{faq}{\small\textgt{このテキストは高校物理の復習ですか?}
... 高校物理と同じ題材も出てきますが, 高校物理では説明され
なかった(説明できなかった)部分を説明し, 理解していきます。
そのために数学をたくさん使います。「高校物理やったから簡単!」と
思ってると, ちょっと痛い目にあうかもね...。}\end{faq}

\begin{faq}{\small\textgt{私は高校物理既習ですが, 有利ではないのですか?}
... 高校物理をやった人は, 有利な面と不利な面があります。それぞれの題材
や問題に慣れ親しんでいるのは有利なこと。高校物理と似た話題を「それ
高校でやった」とスルーしてしまいがちで, 体系的な理解に到達しにくい
のは不利なこと。高校物理を忘れて, 謙虚にやりなおすことが何より大事。}\end{faq}

\begin{faq}{\small\textgt{てことは, もしかして高校物理をやったのは無意味?}
... そんなことありません。人間は成功体験や先入観があると
新しいことを学びづらくなります。それに気をつけよう, と言ってるだけです。
謙虚にやり直せば, 「高校で学んだアレは実はこういうことだったのか!」
という驚きや感動が, 君を大いに成長させてくれるでしょうし, それは
高校で物理をやらなかった人には味わえない体験です。}\end{faq}

\begin{faq}{\small\textgt{簡単な公式は覚えておいた方が良いのでしょうか。} ... 
「公式」よりも, まず「基本法則」と「定義」をしっかり覚えましょう。
「公式」は, 基本法則と定義から論理的に導出されます。}\end{faq}

\begin{faq}{\small\textgt{質問は授業中よりも, 授業後に個別にするほうがいいですよね?} ... 
いえ! 授業中に, その場で質問して下さい!! その方が, 何倍もありがたいです。
「他の人の迷惑」とか「恥ずかしい」とか思わなくていいです。疑問をその場で
ぶつけてくれたら, あなただけでなく他の人たちにも説明できるし, 私の
ケアレスミスもその場で修正できます。}\end{faq}
\hv

2019年3月11日 奈佐原 顕郎
