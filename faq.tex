\section{一問一答}


\begin{faq}{\small\textgt{基礎からこつこつ考えることが重要だと感じた。 / 自分で勉強していくことが大切なのだと思った。} ... だから皆さんには, 大学で4年間という時間が与えられているのです。}\end{faq}\mv

\begin{faq}{\small\textgt{大学の物理は運動方程式を微分積分して色々考察するものだと思っていましたが}
... 1学期の後半は運動方程式がメインです。ただ, 力学は, それだけじゃないんだよね。特に, 
仮想仕事の原理は「最小作用の原理」という考え方に発展し, それは「解析力学」という
物理学に結晶するのです。授業ではやらないけど。}\end{faq}\mv

\begin{faq}{\small\textgt{「$\propto$」とは何ですか? \end{faq}
... 「比例する」です。不明な記号は「数学リメディアル教材」13章を参照。}\end{faq}\mv

\begin{faq}{\small\textgt{パッと見「難しそう」と思ったときあきらめるのではなく一つ一つ丁寧に取り組む姿勢を忘れずにいきたい。} ... 
いい言葉です。ほんとにその通りです。}\end{faq}\mv

\begin{faq}{\small\textgt{微分とか積分とかひとつひとつでは理解したつもりなのに, いざ物理となるとできません。} ... 
大丈夫。きちんと積み上げていれば, いつか, 忽然としてわかるようになります。}\end{faq}\mv

\begin{faq}{\small\textgt{微分方程式の意味を解釈することの重要さを思い知った。 / すっきりした!! / 微分方程式を「読む」
ことが新鮮だった。 / 今日の微分方程式は単純にすごいと思った。式から広がる世界は大きい。/
 微分方程式って聞くとびびってましたが, 克服できそうな気がしました。/ 微分方程式が自分に一生懸命
何かを訴えている感じがしてきました。} ... 
私も, 微分方程式を見ると, 声が聞こえる気がします。「微分方程式は言葉より明快で雄弁だ」という先生もいます。}\end{faq}\mv

%9

\section{一問一答}
\begin{faq}{\small\textgt{外積, なんとなく分かった!} ... 
外積は奥が深いです。3学期の電磁気学でもたくさん使います。}\end{faq}\mv

\begin{faq}{\small\textgt{外積と掛け算マークの見分け方は? } ... 
どちらも$\times$で同じです。両側にベクトルが来れば外積で, スカラーが来れば「掛け算」と判断するしかない。


%11

\section{一問一答}


\begin{itemize}\item 水の中はなぜ無重力ですか? \end{itemize}

比重が1である(密度が水と等しい)ような物体にとっては, 水の中は
無重力です。それは, 重力と浮力が釣り合うからです。浮力はなぜ生じるのか?
それは, 水圧が水深に比例することによります。なぜ水圧は水深に比例するのか?
それは簡単ですから皆さん考えてみてください。\\

\begin{itemize}\item 他に何か参考にすると良い参考書などはありますか? 
又は, 今ある2つ(物理学教材と数学リメディアル教材)で理解するのに充分ですか?\end{itemize}

まずその2つをしっかり勉強する。それでわからなくても, まだ他の本に手を出さないで, 
私のところに質問に来てください。\\

\begin{itemize}\item 高校数学でのベクトルは得意な分野だったので, 
物理学にその考え方が応用できたのがすごい嬉しかったです。\end{itemize}

おめでとう。努力が報われましたね。\\

\begin{itemize}\item 一般式で用いられるアルファベット(万有引力定数$G$ etc)が一気に出て来ると, 
ごっちゃになって式の意味があやふやになってしまいます。これはまず「記号を覚える
$\rightarrow$次元を考えながら丁寧に追って行く」という順序で考えるのがいいですか?\end{itemize}

いろんなアプローチがあると思いますが... そもそも, 覚えねばならない基本法則は
そんなにたくさんはありません。基本法則をしっかり見分けて, 各変数・定数の定義を
しっかり確認することが大事。どういう記号を使うかは, 慣習であって, 本質的ではありません。\\

\begin{itemize}\item 高校までは数学がいちばん理解しづらいと
思っていたけど, 数学より物理が, 物理より化学が理解しづらいということがわ
かりました。\end{itemize}

数学の基礎が無いと総崩れですね。\\


\section{一問一答}
\vspace{0.1cm}
\begin{itemize}\item 物理というものに必要以上にかまえていたが, すんなり理解できてびっくりした。\end{itemize}

物理学は, 明晰で強力で魅力的な学問ですよ。わかることから丁寧に積み上げて
学ぶことが大事です。}\end{faq}\mv


\begin{itemize}\item $g$の定義を$GM/r^2$と答えたらダメなのはなぜですか?\end{itemize}

地表での重力$mg$の$g$は, 点状物体どうしの万有引力の法則の単なる近似ではなく, 
遠心力とか密度の偏りとか, いろんな微妙な要因を飲み込んでいるのです。\\

}\end{faq}\mv


\begin{itemize}\item 電気力がよくわかりません。\end{itemize}

同符号の電荷は反発し, 異符号の電荷は引き合う。その強さはクーロンの法則
にしたがう。とりあえずそれだけ。}\end{faq}\mv


\begin{itemize}\item 化学の授業で角なんとかっていうのが出ました。普通にわかりませんでした。やっぱ
自分で先に化学のために物理を自学しないとダメですか? \end{itemize}

角運動量。高校物理でも範囲外です。今学期の「物理学」の後半にやります。
化学には, なかなか高度な物理が必要ですよ。}\end{faq}\mv






\item 地球の中心からちょっとでもずれた位置に行けば体はバラバラになる? \\
... いえ, 地球中心に向かう重力がかかるだけ。
\item GRACEは, 重いところと軽いところで距離が変化するのはなぜ? \\
... 重いところに近づくと, 先行衛星のほうが後続衛星よりも大きな重力を感じ, その結果, 
先行衛星が加速し, 後続衛星との距離が開く。
\item キャベンディッシュの実験に関する参考書は? \\
... 上記の「ファインマン物理学」I巻にも出ています。
\item 計算の桁をどこまで出せばよいかわかりません。\\
... 問題で与えられた数値の, いちばん少ない桁にあわせればよい。心配ならば, プラスひと桁。
\item オイラーの公式はどのように物理で使われますか? \\
... とりあえず線型応答理論。また, コリオリ力の導出にも便利。本質的には, 
量子力学の波動関数や状態ベクトル。

\item 光速を超えるのは不可能で, 音速を超えると人間はバターのようになってしまう, 
と聞いたことがありますが。\\
... 子供の絵本「ちびくろサンボ」のお話と混同してないか? 

